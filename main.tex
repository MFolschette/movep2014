\documentclass{movep}

\usepackage{hyperref}

\usepackage[english]{babel}
\usepackage[utf8]{inputenc}
\usepackage[T1]{fontenc}

\usepackage{amsmath}  % Maths
\usepackage{amsfonts} % Maths
\usepackage{amssymb}  % Maths
\usepackage{stmaryrd} % Maths (crochets doubles)

\usepackage{enumerate} % Personnalisation de la numérotation des listes
\usepackage{url}     % Mise en forme + liens pour URLs
\usepackage{array}   % Tableaux évolués
\usepackage{comment}

\usepackage{prettyref}
\newrefformat{def}{Definition~\ref{#1}}
\newrefformat{fig}{Figure~\ref{#1}}
\newrefformat{pro}{Property~\ref{#1}}
\newrefformat{pps}{Proposition~\ref{#1}}
\newrefformat{lem}{Lemma~\ref{#1}}
\newrefformat{th}{Theorem~\ref{#1}}
\newrefformat{sec}{Section~\ref{#1}}
%\newrefformat{ssec}{Subsect.~\ref{#1}}
\newrefformat{suppl}{Appendix~\ref{#1}}
\newrefformat{eq}{Eq.~\eqref{#1}}
\def\pref{\prettyref}

%\spnewtheorem*{example*}{Example}{\itshape}{}
\newtheorem*{example*}{Example}{\itshape}{}

\usepackage{tikz}
\newdimen\pgfex
\newdimen\pgfem
\usetikzlibrary{arrows,shapes,shadows,scopes}
\usetikzlibrary{positioning}
\usetikzlibrary{matrix}
\usetikzlibrary{decorations.text}
\usetikzlibrary{decorations.pathmorphing}

%\input{macros/macros}

% Macros générales
\def\Pint{\textsc{PINT}}

% Notations générales pour PH
\newcommand{\PH}{\mathcal{PH}}
\newcommand{\PHs}{\Sigma}
\newcommand{\PHl}{L}
\newcommand{\PHp}{\textcolor{red}{\mathcal{P}}}
\newcommand{\PHproc}{\mathbf{Proc}}
\newcommand{\Proc}{\mathbf{Proc}}
\newcommand{\PHa}{\PHh}
\newcommand{\PHh}{\mathcal{H}}
\newcommand{\PHn}{\mathcal{N}}
\newcommand{\powerset}{\wp}

\newcommand{\PHhitter}{\mathsf{hitter}}
\newcommand{\PHtarget}{\mathsf{target}}
\newcommand{\PHbounce}{\mathsf{bounce}}
\newcommand{\PHsort}{\Sigma}

\def\f#1{\mathsf{#1}}
\def\focals{\f{focals}}
\def\play{\cdot}
\def\configs#1{\mathbb C_{#1\rightarrow a}}
\newcommand{\card}[1]{{|#1|}}

\newcommand{\PHfrappeA}{\rightarrow}
\newcommand{\PHfrappeB}{\Rsh}
\newcommand{\PHfrappe}[3]{#1\PHfrappeA#2\PHfrappeB#3}
\newcommand{\PHfrappebond}[2]{#1\PHfrappeB#2}
\newcommand{\PHobjectif}[2]{\mbox{$#1\PHfrappeB^*\!#2$}}
\newcommand{\PHobj}{\PHobjectif}
\newcommand{\obj}{\PHobjectif}
\newcommand{\PHobjp}[3]{\PHobj{{#1}_{#2}}{{#1}_{#3}}}
\newcommand{\PHconcat}{::}
\newcommand{\PHneutralise}{\rtimes}
\def\Sce{\mathbf{Sce}}
\def\OS{\mathbf{OS}}
\def\Obj{\mathbf{Obj}}
\newcommand{\components}{\Gamma}
\newcommand{\cs}{\Delta}
\def\ctx{\varsigma}
\def\Ctx{\mathbf{Ctx}}
\def\ceil{\f{end}}
\newcommand{\Sol}{\mathbf{Sol}}
\newcommand{\abstr}[1]{#1^\wedge}%\text{\textasciicircum}}
\def\BS{\mathbf{BS}}
\def\aBS{\abstr{\BS}}
\def\abeta{\abstr{\beta}}
\def\aZ{\abstr{\zeta}}
\def\aY{\abstr{\xi}}
\def\indexes#1{\mathbb{I}^{#1}}
\def\emptyseq{\varepsilon}
\def\mconcr{\gamma}
\def\concr{\mconcr_\ctx}
\newcommand{\w}{\omega}
\def\muconcr{\ell}
\def\uconcr{\muconcr_\ctx}

\newcommand{\Pcai}{P(\ctx, a_i)}

\def\PHget#1#2{{#1[#2]}}
\newcommand{\PHchange}[2]{(#1 \Lleftarrow #2)}
\newcommand{\PHarcn}[2]{\mbox{$#1\PHneutralise#2$}}
\newcommand{\PHjoue}{\cdot}
\newcommand{\PHplay}{\PHjoue}

\newcommand{\PHetat}[1]{\mbox{$\langle #1 \rangle$}}

% Notations spécifiques à ce papier
\newcommand{\PHdirectpredec}[1]{\PHs^{-1}(#1)}
\newcommand{\PHpredec}[1]{\f{pred}(#1)}
\newcommand{\PHpredecgene}[1]{\f{reg}({#1})}
\newcommand{\PHpredeccs}[1]{\PHpredec{#1} \setminus \Gamma}

\newcommand{\PHincl}[2]{#1 :: #2}

\def\ctx{\varsigma}
\def\ctxOverride{\Cap}
\def\PHstate#1{\langle #1 \rangle}

\def\DEF{\stackrel{\Delta}=}
\def\EQDEF{\stackrel{\Delta}\Leftrightarrow}

\def\intervalless{<_{[]}}

%\input{macros/macros-ph}

% Notations pour le modèle de Thomas (depuis thèse)
\newcommand{\GRN}{\mathcal{GRN}}
\newcommand{\IG}{\mathcal{G}}
\newcommand{\GRNreg}[1]{\Gamma^{-1}(#1)}
\newcommand{\GRNres}[2]{\mathsf{Res}_{#1}(#2)}
\newcommand{\GRNallres}[1]{\mathsf{Res}_{#1}}
\newcommand{\GRNget}[2]{\PHget{#1}{#2}}
\newcommand{\GRNetat}[1]{\PHetat{#1}}

\def\levels{\mathsf{levels}}
\def\levelsA#1#2{\levels_+(#1\rightarrow #2)}
\def\levelsI#1#2{\levels_-(#1\rightarrow #2)}

\newcommand{\Kinconnu}{\emptyset}
\newcommand{\RRGva}[3]{#1 \stackrel{#2}{\longrightarrow} #3}
\newcommand{\RRGgi}{\mathcal{G}}
\newcommand{\RRGreg}[1]{\RRGgi_{#1}}

\tikzstyle{grn}=[every node/.style={circle,draw=black,outer sep=2pt,minimum
                size=15pt,text=black}, node distance=1.5cm]
\tikzstyle{act}=[->,draw=black,thick,color=black]
\tikzstyle{inh}=[>=|,-|,draw=black,thick, text=black,label]
\tikzstyle{inf}=[->,draw=colorinf,thick,color=colorinf]
\tikzstyle{elabel}=[fill=none,text=black, above=-2pt,%sloped,
minimum size=10pt, outer sep=0, font=\scriptsize, draw=none]
\tikzstyle{sg}=[every node/.style={outer sep=2pt,minimum
                size=15pt,text=black}, node distance=2cm]

%\input{macros/macros-grn}

\usepackage{ifthen}
\usepackage{tikz}
\usetikzlibrary{arrows,shapes}

\definecolor{lightgray}{rgb}{0.8,0.8,0.8}
\definecolor{lightgrey}{rgb}{0.8,0.8,0.8}

\tikzstyle{boxed ph}=[]
\tikzstyle{sort}=[fill=lightgray,rounded corners]
\tikzstyle{process}=[circle,draw,minimum size=15pt,fill=white,
font=\footnotesize,inner sep=1pt]
\tikzstyle{black process}=[process, fill=black,text=white, font=\bfseries]
\tikzstyle{gray process}=[process, draw=black, fill=lightgray]
\tikzstyle{current process}=[process, draw=black, fill=lightgray]
\tikzstyle{process box}=[white,draw=black,rounded corners]
\tikzstyle{tick label}=[font=\footnotesize]
\tikzstyle{tick}=[black,-]%,densely dotted]
\tikzstyle{hit}=[->,>=angle 45]
\tikzstyle{selfhit}=[min distance=30pt,curve to]
\tikzstyle{bounce}=[densely dotted,->,>=latex]
\tikzstyle{hl}=[font=\bfseries,very thick]
\tikzstyle{hl2}=[hl]
\tikzstyle{nohl}=[font=\normalfont,thin]

\newcommand{\currentScope}{}
\newcommand{\currentSort}{}
\newcommand{\currentSortLabel}{}
\newcommand{\currentAlign}{}
\newcommand{\currentSize}{}

\newcounter{la}
\newcommand{\TSetSortLabel}[2]{
  \expandafter\repcommand\expandafter{\csname TUserSort@#1\endcsname}{#2}
}
\newcommand{\TSort}[4]{
  \renewcommand{\currentScope}{#1}
  \renewcommand{\currentSort}{#2}
  \renewcommand{\currentSize}{#3}
  \renewcommand{\currentAlign}{#4}
  \ifcsname TUserSort@\currentSort\endcsname
    \renewcommand{\currentSortLabel}{\csname TUserSort@\currentSort\endcsname}
  \else
    \renewcommand{\currentSortLabel}{\currentSort}
  \fi
  \begin{scope}[shift={\currentScope}]
  \ifthenelse{\equal{\currentAlign}{l}}{
    \filldraw[process box] (-0.5,-0.5) rectangle (0.5,\currentSize-0.5);
    \node[sort] at (-0.2,\currentSize-0.4) {\currentSortLabel};
   }{\ifthenelse{\equal{\currentAlign}{r}}{
     \filldraw[process box] (-0.5,-0.5) rectangle (0.5,\currentSize-0.5);
     \node[sort] at (0.2,\currentSize-0.4) {\currentSortLabel};
   }{
    \filldraw[process box] (-0.5,-0.5) rectangle (\currentSize-0.5,0.5);
    \ifthenelse{\equal{\currentAlign}{t}}{
      \node[sort,anchor=east] at (-0.3,0.2) {\currentSortLabel};
    }{
      \node[sort] at (-0.6,-0.2) {\currentSortLabel};
    }
   }}
  \setcounter{la}{\currentSize}
  \addtocounter{la}{-1}
  \foreach \i in {0,...,\value{la}} {
    \TProc{\i}
  }
  \end{scope}
}

\newcommand{\TTickProc}[2]{ % pos, label
  \ifthenelse{\equal{\currentAlign}{l}}{
    \draw[tick] (-0.6,#1) -- (-0.4,#1);
    \node[tick label, anchor=east] at (-0.55,#1) {#2};
   }{\ifthenelse{\equal{\currentAlign}{r}}{
    \draw[tick] (0.6,#1) -- (0.4,#1);
    \node[tick label, anchor=west] at (0.55,#1) {#2};
   }{
    \ifthenelse{\equal{\currentAlign}{t}}{
      \draw[tick] (#1,0.6) -- (#1,0.4);
      \node[tick label, anchor=south] at (#1,0.55) {#2};
    }{
      \draw[tick] (#1,-0.6) -- (#1,-0.4);
      \node[tick label, anchor=north] at (#1,-0.55) {#2};
    }
   }}
}
\newcommand{\TSetTick}[3]{
  \expandafter\repcommand\expandafter{\csname TUserTick@#1_#2\endcsname}{#3}
}

\newcommand{\myProc}[3]{
  \ifcsname TUserTick@\currentSort_#1\endcsname
    \TTickProc{#1}{\csname TUserTick@\currentSort_#1\endcsname}
  \else
    \TTickProc{#1}{#1}
  \fi
  \ifthenelse{\equal{\currentAlign}{l}\or\equal{\currentAlign}{r}}{
    \node[#2] (\currentSort_#1) at (0,#1) {#3};
  }{
    \node[#2] (\currentSort_#1) at (#1,0) {#3};
  }
}
\newcommand{\TSetProcStyle}[2]{
  \expandafter\repcommand\expandafter{\csname TUserProcStyle@#1\endcsname}{#2}
}
\newcommand{\TProc}[1]{
  \ifcsname TUserProcStyle@\currentSort_#1\endcsname
    \myProc{#1}{\csname TUserProcStyle@\currentSort_#1\endcsname}{}
  \else
    \myProc{#1}{process}{}
  \fi
}

\newcommand{\repcommand}[2]{
  \providecommand{#1}{#2}
  \renewcommand{#1}{#2}
}
\newcommand{\THit}[5]{
  \path[hit] (#1) edge[#2] (#3#4);
  \expandafter\repcommand\expandafter{\csname TBounce@#3@#5\endcsname}{#4}
}
\newcommand{\TBounce}[4]{
  (#1\csname TBounce@#1@#3\endcsname) edge[#2] (#3#4)
}

\newcommand{\TState}[1]{
  \foreach \proc in {#1} {
    \node[current process] (\proc) at (\proc.center) {};
  }
}



\tikzstyle{aS}=[every edge/.style={draw,->,>=stealth}]
\tikzstyle{Asol}=[draw,circle,minimum size=5pt,inner sep=0,node distance=1.5cm]
\tikzstyle{Aproc}=[draw,node distance=1.2cm]
\tikzstyle{Aobj}=[node distance=1.5cm]
\tikzstyle{Anos}=[font=\Large]

%\tikzstyle{AprocPrio}=[Aproc,double]
\tikzstyle{AsolPrio}=[Asol,double]
\tikzstyle{AprocPrio}=[Aproc,double]
\tikzstyle{aSPrio}=[aS,double]



% Commandes À FAIRE
%\usepackage{color} % Couleurs du texte
%\newcommand{\todo}[1]{\textcolor{red}{\textbf{[[#1]]}}}
%\newcommand{\TODO}{\todo{TODO}}


% Id est
%\newcommand{\ie}{\textit{i.e.} }
\newcommand{\ie}{i.e.\ }
\newcommand{\resp}{resp.\ }

% Césures
\hyphenation{pa-ra-me-tri-za-tion}
\hyphenation{pa-ra-me-tri-za-tions}

\title{Efficient analysis on very large models}

\author{%
Maxime Folschette
\institute{LUNAM Universit\'e, \'Ecole Centrale de Nantes, IRCCyN UMR CNRS 6597\\
(Institut de Recherche en Communications et Cybern\'etique de Nantes)\\
1 rue de la No\"e -- B.P. 92101 -- 44321 Nantes Cedex 3, France.}
\email{Maxime.Folschette@irccyn.ec-nantes.fr}
\and
Joint work with: 
Lo\"ic Paulev\'e,
Morgan Magnin,
Olivier Roux
}

\def\titlerunning{Efficient analysis on very large models}
\def\authorrunning{M. Folschette, L. Paulev\'e, M. Magnin, O. Roux}

\begin{document}

\maketitle

\begin{abstract}
The Process Hitting is a recently introduced framework to model concurrent processes.
It models a finite set of components gathering several local states,
with a particular form for the actions.
In this paper, we define the reachability problem that aims at deciding if,
starting from a given initial state, it is possible to activate a given local state.
We also explain the static analysis method that was developed to answer to such problems
in polynomial time, instead of exponential time, thus answering in hundredths of a second
on models with hundreds of components,
but at the price of being sometimes inconclusive.
\end{abstract}



\section{Introduction}

%\todo{Ref PH: \cite{FPIMR12-CMSB}}

Creating a coherent model from an existing system is a very challenging task,
but analyzing may turn out to be even harder.
%Above all problems met, the most important is
%to be able to study it in the end, because it is its main finality.
If we focus on discrete modeling, this kind of problems is recurrent.
In biology, for example, it is rather common to build models
using a series of local experiment in order to determine the interactions between
several components of the system (genes, proteins, metabolites…).
To achieve this, experimenters proceed to gene knockouts
(that is, they prevent some genes to express and stop the production of the related protein)
and then use microarrays to measure the concentration of each other proteins.
%and thus the activity of all genes.
The result of such \textit{in vivo} studies is a model that can gather up to hundreds of components,
but several models of this kind can be also combined into a more general pathway,
which can contain thousands of components.

The problem of being able to efficiently study these models immediately raises.
Formal model checkers usually have to compute the whole dynamics of the model
in order to analyze and confront it to good properties,
often expressed in temporal logics. %, such as in~\cite{Khalis09}.
But the computing of a state graph (even partial) is of exponential complexity,
thus preventing the application of these methods to large models.
%---~that is, above 30 components.
Some results allow to statically derive some results given the structure of the model.
For example, the presence of some circuits in the regulations between genes
%(\ie a circuit with an odd number of inhibitions)
is a necessary condition for interesting behaviors such as oscillations~\cite{RiCo07}.
%is a necessary condition for oscillations in the dynamics of the model~\cite{RiCo07,RRT08}.
%Another approach consists in reducing the state graph \cite{Naldi09}
Although such methods can be of interest, they do not allow precise analysis of the dynamics
in order to validate the functionality of a model or predict \textit{in vivo} behaviors.

In this paper we present an efficient method to compute a specific problem of reachability,
that consists of checking if,
%a given local state of a component can be activated from a
%given initial state.
starting for a given initial state of the system,
a given component can reach a certain local state.
This question is interesting to determine if, fore instance, given a configuration of inputs,
a given output can be activated or not.
The analysis we present here is applied to a specific restriction of synchronous automata networks,
called Process Hitting~\cite{PMR10-TCSB}.
The purpose of this formalism is to model a finite set of local levels,
contained into a finite set of components;
actions are used to change the local level of a component, and can be triggered by
at most one other local level of another component.
Although originally designed for large biological networks,
this formalism is general enough to represent any discrete system.
The particular form of its actions allows an efficient analysis of the
dynamics~\cite{PMR12-MSCS,FPMR13-CS2Bio}
that we present in the following,
which permits to answer reachability questions in polynomial time.
%abstract interpretation that avoids exponential that we present in the following.
%the dynamics of a model can be under- and over-approximated in order to conclude
%on a reachability question in polynomial time, thus avoiding any combinatorial explosion.
The main drawback of this approach is the possibility to be inconclusive
%(if both approximations fail to conclude),
although this case did not occur on the tested examples.

% In \pref{sec:ph} we define the Process Hitting formalism that will be used throughout this paper.
% The reachability problem is defined in \pref{sec:reachability}, along with a general
% presentation of the solving method.
% \pref{sec:glc} defines the graphs of local causality that will be used in
% the two theorems presented in \pref{sec:th}.
% Finally, \pref{sec:examples} gives some benchmark results of the implementation of the method
% presented here, and \pref{sec:conclusion} concludes this paper.



\section{The Process Hitting formalism}
\label{sec:ph}

\subsection{Definition}

\pref{def:PH} introduces the Process Hitting~(PH)~\cite{PMR10-TCSB}
which allows to model
%A Process Hitting (PH) (\pref{def:PH}) gathers 
a finite number of local levels,
called \emph{processes},
grouped into a finite set of components, called \emph{sorts}.
A process is noted $a_i$, where $a$ is the sort's name,
and $i$ is the process identifier within the sort $a$.
At any time, exactly one process of each sort is \emph{active},
and the set of active processes is called a \emph{state}.

The concurrent interactions between processes are defined by a set of \emph{actions}.
Actions describe the replacement of a process by another of the same sort
conditioned by the presence of at most one other process in the current state.
An action is denoted by $\PHfrappe{a_i}{b_j}{b_k}$, which is read as
``$a_i$ \emph{hits} $b_j$ to make it bounce to $b_k$'',
where $a_i,b_j,b_k$ are processes of sorts $a$ and $b$,
called respectively \emph{hitter}, \emph{target} and
\emph{bounce} of the action.
We also call a \emph{self-hit} any action whose hitter and target sorts are the same,
that is, of the form: $\PHfrappe{a_i}{a_i}{a_k}$.

\begin{definition}[Process Hitting]\label{def:PH}
  A \emph{Process Hitting} is a triple $(\PHs,\PHl,\PHa)$:
  \begin{itemize}
    \item $\PHs = \{a,b,\dots\}$ is the finite set of \emph{sorts};
    \item $\PHl = \prod_{a\in\PHs} \PHl_a$ is the set of states with
      $\PHl_a = \{a_0,\dots,a_{l_a}\}$
      the finite set of \emph{processes} of sort $a\in\Sigma$
      and $l_a$ a positive integer, with $a\neq b\Rightarrow \PHl_a \cap \PHl_b = \emptyset$;
    \item $\PHa = \{ \PHfrappe{a_i}{b_j}{b_k} \in \PHl_a \times \PHl_b^2 \mid
      (a,b) \in \PHs^2 \wedge b_j\neq b_k \wedge a=b\Rightarrow a_i=b_j\}$
      is the finite set of \emph{actions}.
  \end{itemize}
\end{definition}

\noindent
The set of all processes is noted: $\Proc = \{ a_i \in \PHl_a \mid a \in \PHs \}$.
Given a state $s\in \PHl$, the process of sort $a\in\PHs$ present in $s$ is denoted by $\PHget{s}{a}$.
An action $h=\PHfrappe{a_i}{b_j}{b_k} \in \PHa$ is \emph{playable} in $s$ if and only if
$\PHget{s}{a}=a_i$ and $\PHget{s}{b} = b_j$.
In such a case, $(s\play h)$ stands for the state resulting from the play of the action
$h$ in $s$, with $\PHget{(s\play h)}{b} = b_k$ and
$\forall c \in \PHs, c \neq b, \PHget{(s\play h)}{c} = \PHget{s}{c}$.
A sequence of actions $\delta = (h_1, \ldots, h_\card{\delta})$ that are successively playable in $s$
is called a \emph{scenario in $s$}, and we note:
$(s \play \delta) = s \play h_1 \play \ldots \play h_\card{\delta}$.
We denote by $\Sce(s)$ the set of scenarios in $s$.

\begin{example*}
\pref{fig:ph} represents a PH $(\PHs,\PHl,\PHa)$ with four sorts
($\PHs = \{a, b, c, d\}$) and:
$\PHl_a = \{a_0, a_1\}$,
$\PHl_b = \{b_0, b_1, b_2\}$,
$\PHl_c = \{c_0, c_1\}$ and
$\PHl_{d} = \{d_0, d_1, d_2\}$.

\begin{figure}[ht]
\centering
\begin{tikzpicture}[font=\scriptsize]
\path[use as bounding box] (-1,-2.9) rectangle (7.8,3.2);
\TSort{(0,0)}{a}{2}{l}
\TSort{(3,0)}{b}{3}{l}
\TSort{(6,0)}{d}{3}{r}
\TSort{(2,-2)}{c}{2}{b}

\THit{a_0}{}{c_0}{.north}{c_1}
\THit{a_1}{}{b_1}{.west}{b_0}
\THit{c_1}{bend left=20pt}{b_0}{.west}{b_1}
\THit{b_1.south west}{->}{a_0}{.east}{a_1}
\THit{b_0}{}{d_0}{.west}{d_1} 
\THit{b_1}{}{d_1}{.west}{d_2} 
\THit{d_1}{}{b_0}{.north east}{b_2}
\THit{c_1}{bend right=80pt,distance=80pt}{d_1}{.east}{d_0}
\THit{b_2}{distance=120pt,out=30,in=40}{d_0}{.east}{d_2}

\path[bounce,bend left]
\TBounce{d_0}{}{d_1}{.south}
\TBounce{d_1}{}{d_2}{.south}
\TBounce{c_0}{}{c_1}{.west}
\TBounce{b_0}{}{b_1}{.south}
\TBounce{d_1}{}{d_0}{.north}
;
\path[bounce,bend right]
\TBounce{a_0}{}{a_1}{.south}
\TBounce{b_0}{}{b_2}{.south}
\TBounce{b_1}{}{b_0}{.north}
\TBounce{d_0}{bend right=50pt,distance=40pt}{d_2}{.south}
;
\TState{a_1,b_0,c_0,d_1}
\node[process] (d_2) at (d_2.center) {{\Large$*$}};

\end{tikzpicture}
\caption{\label{fig:ph}
A PH model example with four sorts.
Circles represent the processes, boxes represent the sorts,
and the actions are drawn by pairs of arrows in solid and dotted lines.
The grayed processes and the asterisk indicate
an example of reachability problem $P(\PHetat{a_1,b_0,c_0,d_1}, d_2)$
as explained in \pref{sec:reachability}.
}
\end{figure}
\end{example*}



\subsection{The reachability problem}
\label{sec:reachability}

%\subsection{Definition}

%First we need to precisely define what “reachability” means in a PH model.
In this paper, we focus on the \emph{reachability of a process}
(\pref{def:reachability}),
which corresponds to the question:
“Is it possible, starting from a given initial state, to play a number of actions
so that a given process is active in the resulting state?”

\begin{definition}[Reachability question]
\label{def:reachability}
  If $\ctx \in \PHl$ is a state and $a_i \in \Proc$ is a process,
  we note $\Pcai$ the \emph{reachability question}:
  $\exists? \delta \in \Sce(s), \PHget{(\ctx \play \delta)}{a} = a_i$.
\end{definition}

In fact, this question can be extended in several manners that will not be tackled here.
The original paper deals with the question of the
\emph{successive reachability of a sequence of processes}~\cite{PMR12-MSCS},
in which several processes are required to be active successively,
which is useful, for example, to state that a component can oscillate.
%, or
%that a process $p_1$ can be reached \emph{even after} the activation of another process $p_2$.
%However this question does not encompass \emph{strict} succession, meaning that
%the $p_2$ could be activated before $p_1$ nonetheless.
Another interesting question %that will not be covered here
is the \emph{simultaneous reachability of several processes};
the addition of priorities in the PH~\cite{FPMR13-CS2Bio},
in addition to increasing the expressivity of the formalism,
allows to easily answer this question.
% This question cannot be tackled only with the static analysis developed here,
% because the PH does not allow to consider simultaneity.
% However, the addition of priorities in the PH,
% besides increasing the expressivity of this formalism,
% permits to answer such questions by using a special sort to represent the
% simultaneous activation of several processes~\cite{FPMR13-CS2Bio}.
Finally, we note that the static analysis developed in the following sections
can also be to some extent generalized to more synchronous automata networks~\cite{PAK13-CAV}.



\subsection{Overview of the solution}

The method developed here relies on the approximation of the dynamics of a PH model,
instead of computing the exact dynamics which requires an exponential time and memory usage
in the size of the model.
Thus, the static analysis we propose
only focuses on local dynamics of the model, that is to say, only computes the dynamics of
single sorts each time it is necessary.
This approximation is based on the fact that an action $\PHfrappe{a_i}{b_j}{b_k}$
can only be triggered by (at most) a process $a_i$ of another sort (in the case where $a_i \neq b_j$).
For example, solving $P(\ctx, b_k)$ would simply require to recursively solve $P(\ctx, a_i)$
if $\PHget{\ctx}{b} = {b_j}$.
In the general case, several actions are needed but the idea is the same.
%which simply multiplies the new reachability questions.
%as solving a reachability question may involve several actions,
%the problem can be recursively solved by considering the reachability questions $P(\ctx, a_i)$
%for all hitters $a_i$ involved in the solving.
This drastically drops the complexity of the method % which now depends only on
%the number of processes in each sort and the number of actions hitting them.
%To be more precise, this method has
to an almost polynomial complexity in the number of sorts in the whole model,
allowing to answer reachability questions in less than a second
on models with hundreds or thousands of sorts.

However, an approximation does not come without drawbacks.
Our method relies on two approximations:
the \emph{under-approximation} $Q$ is a sufficient condition
to answer “Yes” to the reachability question,
and the \emph{over-approximation} $R$ is a necessary condition, used to answer “No”.
Indeed, instead of directly checking a given reachability question $\Pcai$,
we check $Q$ and $R$, which is easier as explained above,
and use the fact that $Q \Rightarrow \Pcai \Rightarrow R$.
But in the case where $Q$ is false and $R$ is true,
the answer turns out to be “Inconclusive”,
and classical model checking techniques have to be used.
Nevertheless, our method has always been conclusive on the examples tested in~\cite{PMR12-MSCS}.



\section{Graphs of local causality}
\label{sec:glc}

The approximations developed in~\cite{PMR12-MSCS,FPMR13-CS2Bio}
rely on the construction of a graph called
\emph{graph of local causality} (GLC),
that permit to approximate the dynamics of a PH model.
One graph is dedicated to the under-approximation, while the other focuses on the over-approximation.
We present an intuitive definition of these graphs in the following,
and illustrate it with the PH model of \pref{fig:ph} and
the reachability question $P(\ctx, d_2)$ where $\ctx = \PHetat{a_1,b_0,c_0,d_1}$.
%, by focusing on a reachability question $\Pcai$.

\subsection{Nodes}

GLCs contain process nodes, objective nodes and solution nodes as detailed below.

\begin{itemize}
  \item
Process nodes (in $\Proc$) allow to state that the reachability of a given process is needed.
\end{itemize}
For example, for the reachability question $P(\ctx, d_2)$, the process $d_2$ will naturally be a node
of the graph.

\emph{Objectives} have the form: $\PHobj{b_j}{b_k}$, which stands for the reachability
of $b_j$ from a state where $b_k$ is present;
it is thereby a refinement of the process node $b_k$ that focuses
on the reachability of a process inside a sort from a given initial process.
We denote the set of all objectives as: $\Obj$.
An objective of the form $\PHobj{b_j}{b_j}$ is called \emph{trivial}
because no action is needed to solve is.
\begin{itemize}
\item
Objective nodes (in $\Obj$) thus permit to name the source and destination processes of a local reachability.
\end{itemize}
For example, $\PHobj{d_1}{d_2}$ denotes the reachability of $d_2$ from
the initial process of $d$ in $\ctx$, which is $d_1$.

Solving an objective $\PHobj{b_j}{b_k}$
implies to find a set of actions that locally solve it.
This can be done by analyzing all bounces and targets of actions on the sort $b$.
and computing the paths without loops that lead from $b_j$ to $b_k$.
%(\ie find a set of actions that locally solve it),
%one has to study the actions hitting processes of $b$.
%Indeed, it is possible to locally solve such an objective by looking at the actions' bounces
%on the sort $b$ and computing the paths without loops that lead from $b_j$ to $b_k$.
Then, if a sequence of actions solving $\PHobj{b_j}{b_k}$ is found,
we abstract it by extracting only the hitters of the actions
because they are the only requirement to play the said actions.
Such an abstracted set of hitters is called a \emph{solution},
and we denote $\Sol = \powerset(\Proc)$ the set of solutions.
A solution indeed abstracts a sequence of actions because only the hitters are kept
and the order of the actions is forgotten (as sets are not ordered).
The solution for trivial objectives,
or objectives that require only self-hits, is the empty set.

\begin{itemize}
  \item 
Solution nodes (in $\Sol$) are sets of processes that are required to locally solve an objective.
\end{itemize}
For example, $\{ c_1, b_2 \}$ and $\{ b_1 \}$ are the two minimal solutions of
the objective $\PHobj{d_0}{d_2}$ in \pref{fig:ph}.

\subsection{Edges}

Edges in a GLC link the different nodes to their requirements.
The GLC of the over-approximation is recursively build the following way:
\begin{itemize}
  \item A process node $b_k$ is linked to the objective of the form: $\PHobj{(\PHget{\ctx}{b})}{b_k}$;
  \item An objective node is linked to all the minimal solutions that solve it;
  \item A solution is linked to all the processes it contains.
\end{itemize}
If a solution is the empty set, then it obviously has no successor, which terminates the recursion.

The under-approximation GLC is built in a similar fashion,
but a process may be linked to several additional objective nodes, in order to become sufficient.
Indeed, if two processes of a sort $b$, say, $b_j$ and $b_k$ are needed to solve
two different objectives, %say $O_1$ and $O_2$,
then we have to ensure that
$b_j$ and $b_k$ are both reachable from the initial state process $\PHget{\ctx}{b}$,
but also that $b_j$ is reachable from $b_k$ and $b_k$ is reachable from $b_j$.
Thus, new objectives $\PHobj{b_j}{b_k}$ and $\PHobj{b_k}{b_j}$ are added,
and this step is executed iteratively until a fixed point is reached.

\begin{comment}
In order to build the GLC of the under-approximation,
we need to extend the notion of initial state $\ctx$ to a more general notion of \emph{context},
in which each sort can contain several (initially) active processes.
This allows to iteratively \emph{complete} this initial context
with all the process nodes contained in the graph.
In other words, this GLC is computed several times (until a fixed point is reached)
by adding in the initial context all the processes that are potentially new source processes
for each objective.
Indeed, if two processes of a sort $b$, say, $b_j$ and $b_k$ are needed to solve
two different objectives, %say $O_1$ and $O_2$,
then one has to ensure that
$b_j$ and $b_k$ are both reachable from the initial state process $\PHget{\ctx}{b}$,
but also that $b_j$ is reachable from $b_k$ and $b_k$ is reachable from $b_j$.
Thus, new objectives $\PHobj{b_j}{b_k}$ and $\PHobj{b_k}{b_j}$ are required,
which is expressed by adding $b_j$ and $b_k$ to the initial context.
Finally, the under-approximation GLC is built the same way
than the over-approximation one, but a process may be linked to several objective nodes,
and the initial context is iteratively expanded as explained.
\end{comment}

We note that the complexity of building these graphs if polynomial in the number of sorts
and exponential in the number of processes in each sort visited
(due to the local solving of objectives in order to find solutions).
However, if the model comprises few processes in every sort
(\ie less that four, which is usual in biological models),
then our method can be considered almost polynomial in the number of sorts, which is very efficient.

\begin{example*}
  \pref{fig:glc} gives two examples of GLCs,
  one for the over-approximation (top) and
  another for the under-approximation (bottom),
  both computed for two different reachability questions on the PH model of \pref{fig:ph}.

\begin{figure}[ht]

\begin{tikzpicture}[aS,node distance=1.25cm]
%\path[draw,use as bounding box] (-0.5,-2.1) rectangle (10.25,1.15);

\node[Aobj] (d12) {$\PHobj{d_1}{d_2}$};
\node[Aproc, above of=d12] (d2) {$d_2$};

\node[Asol,above right of=d12,yshift=-0.4cm] (d12s1) {};
\node[Aproc, right of=d12s1,xshift=1cm] (b2) {$b_2$};
\node[Aobj,right of=b2] (b02) {$\PHobj{b_0}{b_2}$};
\node[Asol,right of=b02] (b02s) {};
\node[Aproc,right of=b02s] (d1) {$d_1$};
\node[Aobj,right of=d1] (d11) {$\PHobj{d_1}{d_1}$};
\node[Asol,right of=d11] (d11s) {};

\node[Asol,below right of=d12,yshift=0.4cm] (d12s2) {};
\node[Aproc, right of=d12s2] (b1) {$b_1$};
\node[Aobj,right of=b1] (b01) {$\PHobj{b_0}{b_1}$};
\node[Asol,right of=b01] (b01s) {};
\node[Aproc,right of=b01s] (c1) {$c_1$};
\node[Aobj,right of=c1] (c01) {$\PHobj{c_0}{c_1}$};
\node[Asol,right of=c01] (c01s) {};
\node[Aproc,right of=c01s] (a0) {$a_0$};
\node[Aobj,right of=a0] (a10) {$\PHobj{a_1}{a_0}$};
\node[xshift=.3cm] at (a10.east) {($\bot$)};

\path
(d12) edge (d12s1) edge (d12s2) (d12s1) edge (b2) edge (c1) (d12s2) edge (b1)
(b01) edge (b01s) (b01s) edge (c1)
(b02) edge (b02s) (b02s) edge (d1)
(c01) edge (c01s) (c01s) edge (a0)
(d11) edge (d11s)
(a0) edge (a10)
(b1) edge (b01)
(b2) edge (b02)
(c1) edge (c01)
(d1) edge (d11)
(d2) edge (d12)
;
\end{tikzpicture}

\vspace*{-.5em}

\hrulefill

\vspace*{.5em}

\begin{tikzpicture}[aS,node distance=1.2cm]

\node[Aobj] (d02) {$\PHobj{d_0}{d_2}$};
\node[Aproc,above of=d02] (d2) {$d_2$};

\node[Asol,right of=d02] (d02s2) {};
\node[Aproc,above right of=d02s2,yshift=-0.2cm] (b0) {$b_0$};
\node[Aobj,right of=b0] (b10) {$\PHobj{b_1}{b_0}$};
\node[Asol,right of=b10] (b10s) {};
\node[Aproc,right of=b10s] (a1) {$a_1$};
\node[Aobj,right of=a1] (a11) {$\PHobj{a_1}{a_1}$};
\node[Asol,right of=a11] (a11s) {};

\node[Aobj,above of=b10,yshift=-0.6cm] (b00) {$\PHobj{b_0}{b_0}$};
\node[Asol,right of=b00] (b00s) {};

\node[Aproc, below of=b0,yshift=0.2cm] (b1) {$b_1$};
\node[Aobj,right of=b1] (b11) {$\PHobj{b_1}{b_1}$};
\node[Asol,right of=b11] (b11s) {};
\node[Aobj,below of=b11,yshift=0.6cm] (b01) {$\PHobj{b_0}{b_1}$};
\node[Asol,right of=b01] (b01s) {};
\node[Aproc,right of=b01s] (c1) {$c_1$};
\node[Aobj,right of=c1] (c11) {$\PHobj{c_1}{c_1}$};
\node[Asol,right of=c11] (c11s) {};

\path
(d02) edge (d02s2) (d02s2) edge (b1) edge (b0)
(a11) edge (a11s)
(b10) edge (b10s) (b10s) edge (a1)
(b11) edge (b11s)
(a1) edge (a11)
(b0) edge (b10) edge (b00.west) (b1) edge (b11)
(d2) edge (d02)
(b00) edge (b00s)
;
\path
(b01) edge (b01s) (b01s) edge (c1)
(c11) edge (c11s)
(b1) edge (b01.west)
(c1) edge (c11)
;
\end{tikzpicture}

\caption{\label{fig:glc}
(top) Over-approximation GLC computed for the PH of \pref{fig:ph}
for $P(\PHetat{a_1,b_0,c_0,d_1}, d_2)$.
(bottom) Under-approximation GLC computed for the PH of \pref{fig:ph}
for $P(\PHetat{a_1,b_1,c_1,d_0}, d_2)$.
}
\end{figure}
\end{example*}



\section{Necessary and sufficient conditions}
\label{sec:th}

Finally, the two computed GLCs can be used in order to conclude on the initial reachability question
$\Pcai$ (\pref{th:oa} and \pref{th:ua}).
Checking these theorems is of polynomial complexity in the number of nodes they contain,
which is limited by the size of the model.

\begin{theorem}[Over-approximation]
\label{th:oa}
  If, starting from the process $a_i$ in the over-approximation GLC,
  there exists no path with no loop such that,
  from an objective node, exactly one linked solution node is traversed
  and, from any other node, all linked nodes are traversed,
  then $\Pcai$ is false.
\end{theorem}

\begin{theorem}[Under-approximation]
\label{th:ua}
  If the under-approximation GLC has no cycle and all its leaves are solution nodes,
  then $\Pcai$ is true.
\end{theorem}

\begin{example*}
  By applying \pref{th:oa} on \pref{fig:glc}(top),
  we can conclude that $P(\PHetat{a_1,b_0,c_0,d_1}, d_2)$ is wrong
  because all paths lead to $\PHobj{a_1}{a_0}$ which has no solution.
  Conversely, with \pref{th:ua} we can conclude that
  $P(\PHetat{a_1,b_1,c_1,d_0}, d_2)$ is true
  because all leafs of on \pref{fig:glc}(bottom) are solutions.
\end{example*}




\section{Application to large models}
\label{sec:examples}

The method presented here was implemented into the
Pint\footnote{The source of Pint and the examples mentioned below are available at
\url{http://loicpauleve.name/pint/}}
library which gathers tools dedicated to PH, and
applied on four biological models as detailed in~\cite{PMR12-MSCS}.
In a nutshell, all models studied have between 42 and 193 sorts,
amongst them what we call \emph{inputs} (\resp \emph{outputs}),
that is, sorts whose processes are not the target (\resp the hitter) of any actions.
For all the possible combinations of the inputs
(that can be either “active” or “inactive”)
the reachability of an “active” state for each output was computed.

The implementation could answer in less than a tenth of a second for each
reachability question.
These computation times were compared to the classical model checkers Biocham~\cite{fages2008formal}
and libDDD~\cite{couvreur2002data}, for which the solving of the same reachability questions
took at least 1 second and in many cases did not terminate due to lack of memory.
Furthermore, our method was always conclusive,
that is, \pref{th:oa} or \pref{th:ua} could always be used in order to conclude,
thus leading to no inconclusive analysis.
Theses examples prove that the approximations developed in the previous sections are very efficient
and allow an unprecedented study of very large models.



\section{Conclusion}
\label{sec:conclusion}

We summarized in this paper the static analysis developed in~\cite{PMR12-MSCS}
and later extended in~\cite{FPMR13-CS2Bio}.
This analysis is applicable to a class of models called Process Hitting that is a
restriction of synchronous automata networks,
and aims at determining if a local state of a component in the model can be attained from a given
initial state.
It consists in approximating the dynamics of the whole model by considering local
reachabilities (objectives)
and sets of local state requirements (solutions) to solve them.
It turns out to be very efficient by making tractable the study of models
with up to hundreds of components,
while classical model checkers take much more time of fail due to memory limitations.



%\bibliographystyle{ieeetr}
%\bibliography{biblio}

\begin{thebibliography}{1}

\bibitem{RiCo07}
A.~Richard and J.-P. Comet, ``Necessary conditions for multistationarity in
  discrete dynamical systems,'' {\em Discrete Applied Mathematics}, vol.~155,
  no.~18, pp.~2403 -- 2413, 2007.

\bibitem{PMR10-TCSB}
L.~{P}aulev{\'e}, M.~{M}agnin, and O.~{R}oux, ``Refining dynamics of gene
  regulatory networks in a stochastic $\pi$-calculus framework,'' in {\em
  Transactions on Computational Systems Biology XIII}, pp.~171--191, Springer,
  2011.

\bibitem{PMR12-MSCS}
L.~Paulev\'{e}, M.~Magnin, and O.~Roux, ``Static analysis of biological
  regulatory networks dynamics using abstract interpretation,'' {\em
  Mathematical Structures in Computer Science}, vol.~22, no.~04, pp.~651--685,
  2012.

\bibitem{FPMR13-CS2Bio}
M.~Folschette, L.~Paulev{\'e}, M.~Magnin, and O.~Roux, ``Under-approximation of
  reachability in multivalued asynchronous networks,'' {\em Electronic Notes in
  Theoretical Computer Science}, vol.~299, pp.~33 -- 51, 2013.
\newblock 4th International Workshop on Interactions between Computer Science
  and Biology (CS2Bio'13).

\bibitem{PAK13-CAV}
L.~Paulev{\'e}, G.~Andrieux, and H.~Koeppl, ``Under-approximating cut sets for
  reachability in large scale automata networks,'' in {\em Computer Aided
  Verification} (N.~Sharygina and H.~Veith, eds.), vol.~8044 of {\em Lecture
  Notes in Computer Science}, pp.~69--84, Springer Berlin Heidelberg, 2013.

\bibitem{fages2008formal}
F.~Fages and S.~Soliman, ``Formal cell biology in {B}iocham,'' in {\em Formal
  Methods for Computational Systems Biology}, pp.~54--80, Springer, 2008.

\bibitem{couvreur2002data}
J.-M. Couvreur, E.~Encrenaz, E.~Paviot-Adet, D.~Poitrenaud, and P.-A.
  Wacrenier, ``Data decision diagrams for {P}etri net analysis,'' in {\em
  Application and Theory of Petri Nets 2002}, vol.~2360 of {\em Lecture Notes
  in Computer Science}, pp.~101--120, Springer, 2002.

\end{thebibliography}

\end{document}
